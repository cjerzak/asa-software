\documentclass[aspectratio=169]{beamer}

% --- Style guide only (no reused content) ---
\usetheme{metropolis}   % Clean, modern theme
\usecolortheme{crane}   % High-contrast palette

% --- Packages ---
\usepackage{graphicx}
\usepackage{tikz}
\usetikzlibrary{arrows.meta,positioning,calc,matrix}
\usepackage{amsmath, amssymb}
\usepackage{booktabs}
\usepackage{hyperref}
\usepackage{mathtools}
\usepackage{csquotes}
\usepackage{comment}
\usepackage{etoolbox}
\usepackage{adjustbox}
\usepackage{multirow}   % for \multirow
\usepackage{array}      % improved column specs
\usepackage{makecell}   % multi-line cells, easier vertical centering

\newcommand\SegCrossEthnicEVUnWt{0}
\newcommand\SegCrossEthnicEVRankUnWt{8}
\newcommand\SegCrossEthnicEVWt{0}
\newcommand\SegCrossEthnicEVRankWt{8}
\newcommand\SegCrossEthnicTP{0}
\newcommand\SegCrossEthnicTPRank{8}
\newcommand\SegCrossEthnicLLM{1}
\newcommand\SegCrossEthnicLLMRank{1}
\newcommand\SegCrossEthnicLP{-9.83}
\newcommand\SegCrossEthnicLPRank{5}
\newcommand\SegDualMajEVUnWt{1}
\newcommand\SegDualMajEVRankUnWt{1}
\newcommand\SegDualMajEVWt{1}
\newcommand\SegDualMajEVRankWt{1}
\newcommand\SegDualMajTP{0.44}
\newcommand\SegDualMajTPRank{2}
\newcommand\SegDualMajLLM{1}
\newcommand\SegDualMajLLMRank{1}
\newcommand\SegDualMajLP{-10.37}
\newcommand\SegDualMajLPRank{6}
\newcommand\SegEthnicEVUnWt{1}
\newcommand\SegEthnicEVRankUnWt{1}
\newcommand\SegEthnicEVWt{1}
\newcommand\SegEthnicEVRankWt{1}
\newcommand\SegEthnicTP{0.66}
\newcommand\SegEthnicTPRank{1}
\newcommand\SegEthnicLLM{1}
\newcommand\SegEthnicLLMRank{1}
\newcommand\SegEthnicLP{-13.44}
\newcommand\SegEthnicLPRank{8}
\newcommand\SegMonoMajEVUnWt{1}
\newcommand\SegMonoMajEVRankUnWt{1}
\newcommand\SegMonoMajEVWt{1}
\newcommand\SegMonoMajEVRankWt{1}
\newcommand\SegMonoMajTP{0.12}
\newcommand\SegMonoMajTPRank{7}
\newcommand\SegMonoMajLLM{1}
\newcommand\SegMonoMajLLMRank{1}
\newcommand\SegMonoMajLP{-4.71}
\newcommand\SegMonoMajLPRank{2}
\newcommand\SegMultiplyLargeEVUnWt{0.94}
\newcommand\SegMultiplyLargeEVRankUnWt{4}
\newcommand\SegMultiplyLargeEVWt{0.98}
\newcommand\SegMultiplyLargeEVRankWt{4}
\newcommand\SegMultiplyLargeTP{0.28}
\newcommand\SegMultiplyLargeTPRank{4}
\newcommand\SegMultiplyLargeLLM{1}
\newcommand\SegMultiplyLargeLLMRank{1}
\newcommand\SegMultiplyLargeLP{-15.07}
\newcommand\SegMultiplyLargeLPRank{9}
\newcommand\SegMultiplySmallEVUnWt{0.94}
\newcommand\SegMultiplySmallEVRankUnWt{4}
\newcommand\SegMultiplySmallEVWt{0.98}
\newcommand\SegMultiplySmallEVRankWt{4}
\newcommand\SegMultiplySmallTP{0.28}
\newcommand\SegMultiplySmallTPRank{4}
\newcommand\SegMultiplySmallLLM{1}
\newcommand\SegMultiplySmallLLMRank{1}
\newcommand\SegMultiplySmallLP{-8.72}
\newcommand\SegMultiplySmallLPRank{4}
\newcommand\SegNonEthnicEVUnWt{}
\newcommand\SegNonEthnicEVRankUnWt{9}
\newcommand\SegNonEthnicEVWt{}
\newcommand\SegNonEthnicEVRankWt{9}
\newcommand\SegNonEthnicTP{0}
\newcommand\SegNonEthnicTPRank{8}
\newcommand\SegNonEthnicLLM{1}
\newcommand\SegNonEthnicLLMRank{1}
\newcommand\SegNonEthnicLP{-2.73}
\newcommand\SegNonEthnicLPRank{1}
\newcommand\SegTwoBigAOneBigBEVUnWt{0.67}
\newcommand\SegTwoBigAOneBigBEVRankUnWt{6}
\newcommand\SegTwoBigAOneBigBEVWt{0.67}
\newcommand\SegTwoBigAOneBigBEVRankWt{6}
\newcommand\SegTwoBigAOneBigBTP{0.44}
\newcommand\SegTwoBigAOneBigBTPRank{2}
\newcommand\SegTwoBigAOneBigBLLM{1}
\newcommand\SegTwoBigAOneBigBLLMRank{1}
\newcommand\SegTwoBigAOneBigBLP{-11.35}
\newcommand\SegTwoBigAOneBigBLPRank{7}
\newcommand\SegTwoMostlyAPartiesSomeBEVUnWt{0.5}
\newcommand\SegTwoMostlyAPartiesSomeBEVRankUnWt{7}
\newcommand\SegTwoMostlyAPartiesSomeBEVWt{0.5}
\newcommand\SegTwoMostlyAPartiesSomeBEVRankWt{7}
\newcommand\SegTwoMostlyAPartiesSomeBTP{0.22}
\newcommand\SegTwoMostlyAPartiesSomeBTPRank{6}
\newcommand\SegTwoMostlyAPartiesSomeBLLM{1}
\newcommand\SegTwoMostlyAPartiesSomeBLLMRank{1}
\newcommand\SegTwoMostlyAPartiesSomeBLP{-7.53}
\newcommand\SegTwoMostlyAPartiesSomeBLPRank{3}
\newcommand\ThreePartyPOneIndex{0.00}
\newcommand\ThreePartyPTwoIndex{0.25}
\newcommand\ThreePartyPThreeIndex{0.75}
\newcommand\ThreePartyLegIndex{0.19}
\newcommand\EthSegIndexA{0.00}
\newcommand\EthSegIndexB{0.67}
\newcommand\EthSegIndexC{0.67}
\newcommand\EthSegIndexD{0.67}
\newcommand\EthSegLegIndex{0.50}

\newcommand{\LLMOverallAccAllpolparty}{0.751}
\newcommand{\LLMOverallAccHighpolparty}{0.860}
\newcommand{\LLMMeanLowConfpolparty}{0.250}
\newcommand{\LLMMedianLowConfpolparty}{0.250}
\newcommand{\LLMMeanAccByCountryHighpolparty}{0.879}
\newcommand{\LLMMeanBaselineByCountrypolparty}{0.536}
\newcommand{\LLMMeanDeltaByCountrypolparty}{0.343}
\newcommand{\LLMMinDeltaByCountrypolparty}{-0.602}
\newcommand{\LLMMaxDeltaByCountrypolparty}{0.761}
\newcommand{\LLMMinDeltaCountrypolparty}{Gambia}
\newcommand{\LLMMaxDeltaCountrypolparty}{Finland}
\newcommand{\LLMMeanAccSmallGroupspolparty}{0.692}
\newcommand{\LLMMeanAccLargeGroupspolparty}{0.820}
\newcommand{\LLMNumCountriespolparty}{114}
\newcommand{\LLMNumInstancespolparty}{34,618}
\newcommand{\LLMNumClassespolparty}{1,209}
\newcommand{\LLMNumPredictionsGainedpolparty}{12,898}
\newcommand{\LLMNumPredictionsGainedPercentpolparty}{20.8}
\newcommand{\LLMRelaxedChangedPctpolparty}{2.5}

\newcommand{\CleavageNumCountries}{153}
\newcommand{\CleavageNumObservations}{368}
\newcommand{\CleavageMinYear}{1990}
\newcommand{\CleavageMaxYear}{2022}
\newcommand{\CleavageOverallMean}{0.077}
\newcommand{\CleavageOverallMedian}{0.020}
\newcommand{\CleavageOverallSD}{0.120}
\newcommand{\CleavageMin}{0.000}
\newcommand{\CleavageMax}{0.667}
\newcommand{\CleavageMeanChange}{-0.049}
\newcommand{\CleavageMedianChange}{-0.001}
\newcommand{\CleavageMeanAbsChange}{0.078}
\newcommand{\CleavageNumChanges}{159}
\newcommand{\CleavageNumCountriesMulti}{141}
\newcommand{\CleavageMeanENP}{2.296}
\newcommand{\CleavageMeanFracGroups}{0.261}
\newcommand{\CleavageMeanUniqueEth}{4.519}
\newcommand{\CleavageMeanAfrica}{0.097}
\newcommand{\CleavageMeanAmericas}{0.064}
\newcommand{\CleavageMeanAsia}{0.070}
\newcommand{\CleavageMeanEurope}{0.067}
\newcommand{\CleavageMeanOceania}{0.134}



% (Optional) biblatex if you want references on the last slide.
% Will only activate if a bib file exists two levels up.
\usepackage[backend=biber,style=authoryear,maxbibnames=99]{biblatex}
\makeatletter
\@ifundefined{IfFileExists}{\newcommand{\IfFileExists}[3]{#2}}{}
\makeatother
\IfFileExists{../../references.bib}{\addbibresource{../../references.bib}}{}
\IfFileExists{../../main.bib}{\addbibresource{../../main.bib}}{}
\IfFileExists{../../paper.bib}{\addbibresource{../../paper.bib}}{}
\IfFileExists{../../mybib.bib}{\addbibresource{../../mybib.bib}}{}

% --- Paths to your paper's figures ---
\graphicspath{{../../FiguresParties/}}

% --- Metropolis tweaks ---
\metroset{progressbar=frametitle, numbering=fraction}
\setbeamertemplate{footline}{}
\setbeamertemplate{caption}[numbered]
\setbeamertemplate{itemize items}[circle]
\setbeamertemplate{section in toc}[sections numbered]

% --- Convenience macros (kept minimal) ---
\newcommand{\cleav}{\mathsf{Cleavage}~\mathsf{Index}}
\newcommand{\E}{\mathbb{E}}
\newcommand{\R}{\mathbf{R}}
\newcommand{\Omat}{\mathbf{O}}

% --- Title block ---
\title{Elite Cleavages:\\ Concept, Measurement, and a Verifiable AI Search Agent}
\author{\textbf{John Gerring} \and \textbf{Connor T. Jerzak} \and \textbf{Erzen Öncel}}
\institute{University of Texas at Austin \quad\textbullet\quad Özyeğin University}
\date{APSA 2025 \textbullet\ September 2025}

% --- Section headers as mini-tour slides ---
\AtBeginSection{
  \begin{frame}[standout]
    \Large \insertsection
  \end{frame}
}

\begin{document}

% --------------------------------------------------
\maketitle
% --------------------------------------------------

\begin{frame}{Elite Cleavages: What are they and why they matter}{}
\large 
\begin{itemize}
  \item The role of ethnicity in politics isn't only about descriptive representation -- it's also about how \emph{elite party delegations} {\bf reproduce} social boundaries.
  \item We lack \textbf{continuous, cross-national} metrics at the elite level.
  \item Our solution: a \textbf{single, bounded, interpretable} index from optimal transport — plus \textbf{party-} and \textbf{group-level} diagnostics.
  \begin{itemize}
     
  \end{itemize}
\end{itemize}
\end{frame}

% --------------------------------------------------
\section{Measuring elite cleavages}
% --------------------------------------------------
\begin{frame}{Ethnic cleavages: Our approach}
\begin{itemize}
\begin{enumerate}
\item Centered on elites \textit{(representatives)} rather than masses \textit{(voters)}.
\item Treats cleavages as matters of degree.
\item Assigns scores to (a) individual parties, (b) ethnic groups, as well as (c) polities (the overall cleavage).
\item Applicable to any setting—local, regional, or national—where the party and ethnic identity of MPs can be ascertained.
\end{enumerate}
\end{itemize}
\end{frame}

\begin{frame}{From idea to index: Scenarios clarify}
\begin{table}[htb]
\caption{Illustrative scenarios.}\label{tab:Illustrative}
\centering\scriptsize
\begin{tabular}{l | ccc | ccc | ccc | ccc | ccc | ccc  }
\toprule
  {\it Legislatures} & \multicolumn{3}{c|}{\bf  I } & \multicolumn{3}{c|}{\bf  II } & \multicolumn{3}{c|}{\bf  III } & \multicolumn{3}{c|}{\bf  IV } & \multicolumn{3}{c|}{\bf  V } & \multicolumn{3}{c}{\bf  VI }  \\
%
%\cmidrule(lr){2-4} \cmidrule(lr){5-7} \cmidrule(lr){8-10} \cmidrule(lr){11-13} \cmidrule(lr){17-19} \cmidrule(lr){20-22} 
%
$(N)$ & \multicolumn{3}{c|}{(12)} & \multicolumn{3}{c|}{(12)} & \multicolumn{3}{c|}{(12)} & \multicolumn{3}{c|}{(12)} &  \multicolumn{3}{c|}{(14)} & \multicolumn{3}{c}{(12)} \\
\midrule
{\it Ethnic groups} & 
\multicolumn{3}{c|}{{\bf A}} & 
{\bf A} & {\bf B} & {\bf C} &
\multicolumn{3}{c|}{\bf A \quad \bf B} 
& {\bf A} & {\bf B} & {\bf C}  & 
{\bf A} & {\bf B} & {\bf C} & 
{\bf A} & {\bf B} & {\bf C}  \\
%
$(N)$ & 
\multicolumn{3}{c|}{(12)} &
(4) & (4) & (4) & 
\multicolumn{3}{c|}{ (8) \quad  (4)} &
(10) & (1) & (1) & 
(8) & (4) & (2) & 
(4) & (4) & (4)  \\
\midrule
{\it Parties} & {\bf P1} & {\bf P2} & {\bf P3} & {\bf P1} & {\bf P2} & {\bf P3} & {\bf P1} & {\bf P2} & {\bf P3} & {\bf P1} & {\bf P2} & {\bf P3} & {\bf P1} & {\bf P2} & {\bf P3} & {\bf P1} & {\bf P2} & {\bf P3}  \\
%
$(N)$ & (4) & (4) & (4) & (4) & (4) & (4) & (6) & (4) & (2) & (5) & (5) & (2) & (4) & (4) & (6) & (4) & (4) & (4) \\
%
%Party Seg. & \SegNonEthnicP1{} & \SegNonEthnicP2{} & \SegNonEthnicP3{} & \SegCrossEthnicP1{} & \SegCrossEthnicP2{} & \SegCrossEthnicP3{} & \SegIVP1{} & \SegIVP2{} & \SegIVP3{} & \SegVIP1{} & \SegVIP2{} & \SegVIP3{} &  &  &  & \SegVP1{} & \SegVP2{} & \SegVP3{} & \SegEthnicP1{} & \SegEthnicP2{} & \SegEthnicP3{} &  &  &  \\
\addlinespace
\multirow{8}{}{\vspace{1.75cm}\textit{MPs}} & {\it a} & {\it a} & {\it a} & {\it a} & {\it a} & {\it a} & {\it a} & {\it a} & {\it b} & {\it a} & {\it a} & {\it b}  & {\it a} & {\it a} & {\it b} & {\it a} & {\it b} & {\it c} \\
 & {\it a} & {\it a} & {\it a} & {\it b} & {\it b} & {\it b} & {\it a} & {\it a} & {\it b} & {\it a} & {\it a} & {\it c} & {\it a} & {\it a} & {\it b} & {\it a} & {\it b} & {\it c} \\
 & {\it a} & {\it a} & {\it a} & {\it c} & {\it c} & {\it c} & {\it a} & {\it a} &  & {\it a} & {\it a} &  & {\it a} & {\it a} & {\it b} & {\it a} & {\it b} & {\it c} \\
 & {\it a} & {\it a} & {\it a} & {\it d} & {\it d} & {\it d} & {\it a} & {\it a} &  & {\it a} & {\it a} & {\it } & {\it a} & {\it a} & {\it b} & {\it a} & {\it b} & {\it c}  \\
 &  &  &  &  &  &  & {\it b} &  &  & {\it a} & {\it a} &  &  &  & {\it c} &  &  &  \\
 &  &  &  &  &  &  & {\it b} &  &  &  &  &  &  &  & {\it c} &  &  &  \\
  \midrule
 %
\textit{Leg. Seg. Index} &
  \multicolumn{3}{c|}{\SegNonEthnicTP{} } &
  \multicolumn{3}{c|}{\SegCrossEthnicTP{} } &
  \multicolumn{3}{c|}{\SegTwoMostlyAPartiesSomeBTP{} } &
  \multicolumn{3}{c|}{\SegMultiplySmallTP{} } &
  \multicolumn{3}{c|}{\SegTwoBigAOneBigBTP{} } &
  \multicolumn{3}{c}{\SegEthnicTP{} }  \\
%
%Group seg. & \multicolumn{3}{c|}{0} & 0 & 0 & 0 & \multicolumn{3}{c|}{} & \multicolumn{3}{c}{} & \multicolumn{3}{c|}{} & \multicolumn{3}{c|}{} & $\sim$1 & $\sim$1 & $\sim$1 \\
\bottomrule
\end{tabular}
\end{table}
\begin{center}
Less elite cleavage $\Leftrightarrow$ more elite cleavage
\end{center}
\end{frame}

\begin{frame}{From idea to index: Insights from Optimal Transport}{}
\large
\begin{itemize}
  \item Think of the legislature as a contingency table: \emph{groups} $\times$ \emph{parties}.
  \item Think of the non-cleavage exemplar: \textbf{perfect integration}, where every party mirrors the chamber’s overall group composition.
  \item \textbf{Cleavage} = \underline{fewest seat reassignments} needed to reach perfect integration (normalized)
  \item Interpretable scale in $[0,1)$: higher $\Rightarrow$ more ethnicized elite alignment.
  \begin{itemize}
\item Optimal transport approach $\leadsto$ extendable to \textbf{party} and \textbf{group} level diagnostics
  \end{itemize}
\end{itemize}
\end{frame}

\begin{frame}{From idea to index: Optimal Transport, visualized}
\vspace{-2.cm}
\small
\centering

\begin{tikzpicture}[x=1cm,y=1cm,>=Latex]
% ---------- layout & styles ----------
\def\cellw{1.20}   % cell width (cm)
\def\cellh{0.90}   % cell height (cm)
\def\xL{0.0}       % left matrix x origin
\def\xR{5.8}       % right matrix x origin
\def\yTop{2.2}     % top row y coordinate

\tikzset{
  cell/.style={draw, minimum width=\cellw cm, minimum height=\cellh cm, align=center},
  head/.style={font=\scriptsize\bfseries},
  lab/.style={font=\scriptsize},
  flow/.style={-Latex, thick, draw=alerted text.fg},
  box/.style={draw=gray!60, rounded corners, fill=gray!10}
}

% ---------- titles ----------
\node[head] at (\xL+1.8, \yTop+1.05) {Observed $\Omat$};
\node[head] at (\xR+1.9, \yTop+1.05) {Perfect integration $\R$};

% ---------- column headers ----------
\foreach \j/\name in {0/P1,1/P2,2/P3}{
  \node[head] at (\xL+\j*\cellw, \yTop+0.45) {\name};
  \node[head] at (\xR+\j*\cellw, \yTop+0.45) {\name};
}

% ---------- row headers ----------
\foreach \i/\g in {0/Group 1,1/Group 2,2/Group 3}{
  \node[head,anchor=east] at (\xL-0.25, \yTop-\i*\cellh) {\g};
}

% ---------- observed cells (numbers sum to n=60) ----------
% Row 1 (G1): [24, 5, 1]
\node[cell] (O11) at (\xL+0*\cellw, \yTop-0*\cellh) {24};
\node[cell] (O12) at (\xL+1*\cellw, \yTop-0*\cellh) {5};
\node[cell] (O13) at (\xL+2*\cellw, \yTop-0*\cellh) {1};
% Row 2 (G2): [5, 10, 3]
\node[cell] (O21) at (\xL+0*\cellw, \yTop-1*\cellh) {5};
\node[cell] (O22) at (\xL+1*\cellw, \yTop-1*\cellh) {10};
\node[cell] (O23) at (\xL+2*\cellw, \yTop-1*\cellh) {3};
% Row 3 (G3): [1, 5, 6]
\node[cell] (O31) at (\xL+0*\cellw, \yTop-2*\cellh) {1};
\node[cell] (O32) at (\xL+1*\cellw, \yTop-2*\cellh) {5};
\node[cell] (O33) at (\xL+2*\cellw, \yTop-2*\cellh) {6};

% ---------- target cells (perfect integration: each party mirrors chamber) ----------
% Totals: Party totals (30, 20, 10), Group totals (30, 18, 12) → \R =
% G1: [15,10,5], G2: [9,6,3], G3: [6,4,2]
\visible<2->{
  \node[cell] (R11) at (\xR+0*\cellw, \yTop-0*\cellh) {15};
  \node[cell] (R12) at (\xR+1*\cellw, \yTop-0*\cellh) {10};
  \node[cell] (R13) at (\xR+2*\cellw, \yTop-0*\cellh) {5};

  \node[cell] (R21) at (\xR+0*\cellw, \yTop-1*\cellh) {9};
  \node[cell] (R22) at (\xR+1*\cellw, \yTop-1*\cellh) {6};
  \node[cell] (R23) at (\xR+2*\cellw, \yTop-1*\cellh) {3};

  \node[cell] (R31) at (\xR+0*\cellw, \yTop-2*\cellh) {6};
  \node[cell] (R32) at (\xR+1*\cellw, \yTop-2*\cellh) {4};
  \node[cell] (R33) at (\xR+2*\cellw, \yTop-2*\cellh) {2};
}

% ---------- minimal within-group flows (OT plan): show only 1–2 illustrative arrows ----------
\visible<3->{
  % Example 1 (G1): move 5 from P1 -> P2
  \draw[flow] (O11.east) to[bend left=12] node[lab, above] {5} (R12.west);

  % Example 2 (G3): move 4 from P3 -> P1
  \draw[flow] (O33.east) to[bend left=8] node[lab, above] {4} (R31.west);
}

% ---------- separation hint ----------
\draw[densely dashed, gray!50] ($(O13.east)+(0.7,1.0)$) -- ($(O33.east)+(0.7,-1.0)$);

% ---------- formula + interpretation (compact, with breathing room) ----------
\visible<4->{
  \node[box, align=left, font=\scriptsize, anchor=west, inner sep=1.5pt] (form)
        at (\xL+0.10, \yTop-3.50) {%
    \(\displaystyle W=\tfrac12\sum_{i,j}\!\left|o_{ij}-r_{ij}\right|=18,\; n=60,\quad
      S=W/n=0.30\in[0,1)\). {\footnotesize Higher $S$ $\Rightarrow$ more cleavage.}
  };
}

% ---------- footnote ----------
\node[align=center, font=\tiny, text width=0.9\linewidth] at (\xL+3.9, -0.5) {%
  Minimal reassignments are computed \emph{within rows} (ethnicity fixed); only party switches are allowed.
  The right panel is the perfect-integration target where each party mirrors the chamber’s group composition.
};

\end{tikzpicture}

% ---------- overlay guide ----------
\medskip
\onslide<1->{\footnotesize \textit{(1) Observed}} \hfill
\onslide<2->{\footnotesize \textit{(2) Target}} \hfill
\onslide<3->{\footnotesize \textit{(3) Example flows}} \hfill
\onslide<4->{\footnotesize \textit{(4) Index $S=W/n$}}
\end{frame}


% --------------------------------------------------
\section{Data \& Coding}
% --------------------------------------------------

\begin{frame}{Base data: Expert codings}
\large
\begin{itemize}
  \item Base data $\leadsto$ legislative delegations assembled from the \textbf{Global Leadership Project} (GLP).
  \begin{itemize}
  \item \textbf{Expert coders} identify salient ascriptive groups (country-specific).
  \item Focus on \textbf{represented} groups; aim is alignment among elites, not overall proportionality.
  \end{itemize}
\end{itemize}
\end{frame}

\begin{frame}{Augmented data: A verifiable AI search agent}
\begin{columns}[T,onlytextwidth]
\column{0.58\textwidth}
\large
\begin{itemize}
  \item \textbf{Goal}: Reduce substantive missingness in global scale data on party membership (good starting point$\leadsto$ objective)
  \item \textbf{Closed party codebooks} by country; agent predicts or \emph{abstains}, providing confidence scores also.
  \item \textbf{Sources:} Agent searches parliamentary portals, official bios, reputable news, encyclopedia entries (can re-try search 20 times).
  \item \textbf{Provenance preserved:} queries, snippets, links archived for audit and replication.
\end{itemize}
\column{0.48\textwidth}
\centering
\includegraphics[width=\linewidth]{AgentViz.pdf}
\end{columns}
\end{frame}

\begin{frame}{How did the agent do?}
\begin{figure}[htb]
    \centering
\includegraphics[width=0.55\linewidth]{FiguresParties/AgentOverTime.pdf}
\includegraphics[width=0.43\linewidth]{FiguresParties/AgentRegionBox.pdf}
\caption{
Agent performance over time; point size corresponds to the number of observations in each (approximate) year (\textsc{Left}) and space (\textsc{Right}). 
\\ \textbf{Overall:} 90\% median country accuracy (40\% upload over baseline)
}
\label{fig:AgentsOverTime}
\end{figure}
\end{frame}

% --------------------------------------------------
\section{Patterns}
% --------------------------------------------------

\begin{frame}{Global distribution}
\centering
\begin{minipage}{0.49\linewidth}
\centering
\includegraphics[width=\linewidth]{Histogram_CountryLevel_CleavageIndex.pdf}
\end{minipage}\hfill
\begin{minipage}{0.49\linewidth}
\centering
\includegraphics[width=\linewidth]{Boxplot_Region_CleavageIndex.pdf}
\end{minipage}

\vspace{0.7em}
\large
\begin{itemize}
  \item Right-skewed: most legislatures cluster near low cleavage; long tail of segmented systems.
  \item Regional heterogeneity is substantial.
\end{itemize}
\end{frame}

\begin{frame}{Stability over time}
\centering
\includegraphics[width=0.85\linewidth]{Trajectories_Country.pdf}

\vspace{0.6em}
\large
\begin{itemize}
  \item Within-country trajectories are typically \textbf{persistent}, with gradual drifts or mild oscillations.
\end{itemize}
\end{frame}

% --------------------------------------------------
\section{Interpretation}
% --------------------------------------------------

\begin{frame}{What the regressions say: Simple supply/demand story explains most variability}
\large
\begin{itemize}
\item \textbf{Compositional Factors}: 
\item[]  $\leadsto$ Ethnic \& party fractionalization (strongest predictors)
\item[] $\leadsto$ Simple interaction model explains 74\% of variability
\item[] $\leadsto$ When both party and group fractionalization are high, elite cleavage is generally much more pronounced 
  \item Sociological, economic, and demographic moderators show intuitive but less stable associations than compositional factors.
\end{itemize}
\end{frame}

\begin{frame}{Outcome: Predicting elite cleavage}{}
\vspace{-0.2cm}
{
\begingroup
\tiny
% keep everything tiny even if the input uses \small, \footnotesize, etc.
\let\small\tiny
\let\footnotesize\tiny
\let\scriptsize\tiny
\let\normalsize\tiny

% Table created by stargazer v.5.2.3 by Marek Hlavac, Social Policy Institute. E-mail: marek.hlavac at gmail.com
% Date and time: Tue, Oct 07, 2025 - 12:17:06
\begin{table}[htbp] \centering 
\footnotesize 
\begin{tabular}{@{\extracolsep{5pt}} lcccc} 
\\[-1.8ex]\hline 
\hline \\[-1.8ex] 
 & Model 1 & Model 2 & Model 3 & Model 4 \\ 
\hline \\[-1.8ex] 
Party Fractionalization & 0.15 (7.94)$^*$  & 0.01 (1.04) & 0.03 (1.25) & 0.02 (1.83) \\ 
Group Fractionalization & 0.27 (12.06)$^*$  & 0.01 (0.92) & 0.01 (0.30) & 0.00 (0.26) \\ 
Party Times Group Fractionalization &  & 0.63 (12.94)$^*$  & 0.55 (6.03)$^*$  & 0.63 (12.40)$^*$  \\ 
  &  &  &  &  \\ 
Polyarchy Index &  &  &  & 0.03 (1.18) \\ 
Power Distributed by Social Group &  &  &  & -0.01 (-1.64) \\ 
Access to Public Services by Social Group &  &  &  & 0.00 (-0.32) \\ 
Access to State Jobs by Social Group &  &  &  & 0.00 (-0.20) \\ 
Access to State Business Opportunities by Social Group &  &  &  & 0.00 (0.02) \\ 
  &  &  &  &  \\ 
log(Population) &  &  &  & 0.00 (-1.54) \\ 
log(GDP per capita (PPP)) &  &  &  & -0.01 (-1.14) \\ 
Country, Percent of Pop. Urbanized &  &  &  & 0.00 (0.60) \\ 
  &  &  &  &  \\ 
Country FE &  &  & \checkmark &  \\ 
Year FE &  &  & \checkmark &  \\ 
  &  &  &  &  \\ 
\emph{Other statistics} &  &  &  &  \\ 
Countries & 153 & 153 & 153 & 139 \\ 
Observations & 378 & 378 & 378 & 341 \\ 
Adjusted R-squared & 0.57 & 0.74 & 0.77 & 0.74 \\ 
\hline \\[-1.8ex] 
\end{tabular} 
  \caption{Outcome: Cleavage Index. Estimator: 
                  OLS with clustered standard errors by country. 
                  $*$ indicates $p$ < 0.05; $t$-statistics are in parentheses. } 
  \label{tab:RegEVTab_ExpertsAndAgent_SEanalytical} 
\end{table} 

\endgroup
}
\end{frame}


\begin{frame}{Conclusion \& future work}{}
{\sc To sum up, our contributions:}
\begin{itemize}
  \item[] \textbf{Measurement:} Elite cleavage as minimal reassignments to perfect integration.
  \item[] \textbf{Scope:} Global coverage via expert coding + a \emph{verifiable} AI search agent.
  \item[] \textbf{Findings:} Cleavages co-move with social/party heterogeneity \emph{amplifying} demand.
  \item[] \textbf{Tools:} Party-/group-level diagnostics; extendable to other ascriptive cleavages.
\end{itemize}
{\sc Future work:}
\begin{enumerate}
  \item \textbf{Ethnicity is contextual:} relies on country-specific codebooks.
  \item We focus on \textbf{represented} groups; not a test of proportionality in society.
\end{enumerate}
%{\sc Implications:}
%\begin{itemize}
%\item A \textbf{replicable baseline} for cross-national/over-time comparisons of elite segmentation.
 % \item \textbf{Diagnostics} pinpoint where segmentation sits (parties vs.\ groups).
%  \item Enables targeted evaluation of \textbf{reforms} (electoral rules).
  %\item Extends naturally to \textbf{religion, language, region}, and other arenas (cabinets).
%\end{itemize}
\vspace{-.2cm}
\textbf{For more information:} \texttt{{\color{orange}Global}{\color{blue}Leadership}{\color{gray}Project}.net}
\end{frame}

%\begin{frame}[standout]
%\Large Takeaway: Elite ethnic cleavages\\ are measurable, comparable, and interpretable.
%\end{frame}

\begin{comment}

% --------------------------------------------------
\section{Supplementary information}
% --------------------------------------------------

\begin{frame}{Formal details (index)}
\small
Observed matrix $\Omat=[o_{ij}]$ and target $\R=[r_{ij}]$ with $r_{ij}=(o_{i+}o_{+j})/n$ preserve margins. With a 0–1 party reassignment cost (same party $=0$, different party $=1$) and ethnicity fixed, the transport objective reduces to total variation:
\[
W(\Omat,\R) = \frac{1}{2}\sum_{i}\sum_{j}\left|o_{ij}-r_{ij}\right|,\qquad
S=\frac{W(\Omat,\R)}{n}\in[0,1).
\]
\textit{Interpretation:} \(S\) is the minimal \emph{share of seats} that must move across parties (within groups) to achieve perfect integration.
\end{frame}

\begin{frame}{Party- and group-level indices}
\small
\begin{itemize}
  \item \textbf{Party index:} difference between a party’s delegation and chamber composition.
  \item \textbf{Group cohesion:} dispersion vs.\ concentration across parties (same transport logic).
  \item \textbf{Aggregation:} chamber-level cleavage is a coherent summary of these components.
\end{itemize}
\end{frame}

\begin{frame}{Agent prompts (distilled) — verifiable coding}
\small
\begin{itemize}
  \item Country-specific \textbf{closed codebook} of allowable labels.
  \item \textbf{Retrieve \& cite:} parliament portals, official bios, reputable media, encyclopedia.
  \item \textbf{Return:} (label, one-sentence rationale, source links). \textbf{Abstain} if evidence weak/conflicting.
  \item \textbf{Archive all} queries/snippets/links for audit and replication.
\end{itemize}
\end{frame}

\begin{frame}{Related measures (for context)}
\small
\begin{itemize}
  \item Binary “ethnic party” codings: broad coverage; \emph{loss of granularity}.
  \item Survey-based ethnic voting distances: mass behavior focus; \emph{not} elite composition.
  \item Our transport index: continuous, elite-focused, party/group diagnostics, \textbf{bounded \& interpretable}.
\end{itemize}
\end{frame}

% Optional bibliography slide (only if a .bib was found above)
\IfFileExists{../../references.bib}{%
\begin{frame}[allowframebreaks]{References}
\printbibliography
\end{frame}
}{}
\IfFileExists{../../main.bib}{%
\begin{frame}[allowframebreaks]{References}
\printbibliography
\end{frame}
}{}
\IfFileExists{../../paper.bib}{%
\begin{frame}[allowframebreaks]{References}
\printbibliography
\end{frame}
}{}

\begin{frame}{Formalisms}
\large
Let $\Omat = [o_{ij}]$ be observed counts of group $i$ in party $j$, with totals $n$, row sums $o_{i+}$, and column sums $o_{+j}$. Define the \emph{integration target}
\[
  \R = [r_{ij}],\quad r_{ij} = \frac{o_{i+}\,o_{+j}}{n}.
\]
Costs: reassigning across parties costs $1$; changing ethnicity is forbidden (infinite cost). The cleavage distance is
\[
  W(\Omat,\R) ~=~ \frac{1}{2}\sum_{i}\sum_{j}\big|o_{ij}-r_{ij}\big|.
\]
\textbf{Index:} \(
  S ~=~ \dfrac{W(\Omat,\R)}{n}
\) \;\;=\;\; minimal \textit{share of seats} needing reassignment to achieve integration.
\end{frame}

\begin{frame}{Beyond the system: party \& group diagnostics}
\large
\begin{itemize}
  \item \textbf{Party-level “ethnic party” index:} how far a party is from mirroring the chamber.
  \item \textbf{Group-level “cohesion” index:} concentration of a group within a party.
  \item All derived from the same transport logic, mapping segmentation across scale.
\end{itemize}
\end{frame}

\begin{frame}{Demand meets supply?}
\centering
\includegraphics[width=0.78\linewidth]{hypothesis_test_heatmap.png}

\vspace{0.6em}
\large
\begin{itemize}
  \item Cleavage rises with \textbf{social heterogeneity} and is \emph{amplified} when party systems are more fragmented.
  \item High cleavage concentrates where both are large.
\end{itemize}
\end{frame}

\begin{frame}{Party system fragmentation and cleavage}
\centering
\includegraphics[width=0.5\linewidth]{NPartiesEff_vs_EthnicParties.pdf}

\vspace{0.6em}
\large
\begin{itemize}
  \item Positive association overall; attenuation once \textbf{demand} is controlled.
\end{itemize}
\end{frame}
\end{comment}

\end{document}
